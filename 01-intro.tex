\section{Introduction}

%% \jbs{I expect this entire paper to be about 3-4 pages. We should be clear
%% and concise, but still put all our ideas down.}

Instruction caches are widely used to mediate the effects of reads
from main memory, relative to computation time. 
The instruction cache is accessed for every instruction executed and
program execution time can vary widely depending on the number of
instruction cache misses~\cite{arnold94}. 
In existing Graphics Processing Units (GPUs) and CPUs, each processor
core has its own instruction cache. 
A unified or shared instruction cache that is used by all, or many
cores of a GPU or CPU has the potential to improve system performance
and reduce power consumption.
However, such a modification also results in increased traffic for the
instruction cache, which could lead to a higher miss rate, reducing
performance and increasing power consumption. 
In current computer architectures, the instruction caches are
independent of one another, however since each processor uses the same
set of instructions, it is plausible that a shared instruction cache
could introduce non-trivial improvements in performance. 
Unified instruction cache(s) could reduce the number of compulsory
misses because an instruction previously executed by one streaming
multiprocessor may be available for another streaming multiprocessor
immediately rather than requiring an additional miss. 
In addition to a fully unified instruction cache used across all
processor cores, another possible solution could be to maintain
multiple instruction caches, which are shared across a subset of all
cores. 
This architecture could allow the operating system to group together
program threads based on similarity of instructions to maximize the
benefit of shared instructions and minimized the conflicts across
threads. 

