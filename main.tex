\documentclass[pageno]{jpaper}
\pdfoutput=1
\usepackage[normalem]{ulem}
\usepackage{amsmath}

% Comment commands - easy to remove
\usepackage{color}
\definecolor{orange}{rgb}{1,0.5,0}
\newcommand{\jbs}[1]{{\color{blue}[\textbf{\sc JBS}: \textit{#1}]}} %Josef
\newcommand{\dhp}[1]{{\color{red}[\textbf{\sc DHP}: \textit{#1}]}} %Dong-hyeon
\newcommand{\ab}[1]{{\color{magenta}[\textbf{\sc AB}: \textit{#1}]}}%Akhil
\newcommand{\fh}[1]{{\color{orange}[\textbf{\sc FH}: \textit{#1}]}}  %Fabiha
\newcommand{\es}[1]{{\color{Green}[\textbf{\sc ES}: \textit{#1}]}}  %Eric

% graphics path
\graphicspath{ {./graphics/} }

\begin{document}

\title{Sphynx: A Shared Instruction Cache Exporatory Study}

\author{\vspace{.08in} Dong-hyeon Park \and Akhil Bagaria \and Fabiha Hannan \and
  Eric Storm \and Josef Spjut}

\date{}
\maketitle

\thispagestyle{empty}

\begin{abstract}
This document is intended to serve as a sample for submissions to MICRO 2013.
We provide some guidelines that authors should follow when submitting papers to
the conference.

\end{abstract}

\section{Introduction}

\jbs{I expect this entire paper to be about 3-4 pages. We should be clear
and concise, but still put all our ideas down.}

Instruction caches are widely used to mediate the effects of reads from main memory, relative to computation time. 
The instruction cache is accessed for every instruction executed and program execution time can vary widely depending on the number of instruction cache misses~\cite{arnold94}. 
In existing Graphics Processing Units (GPUs) and CPUs, each processor core has its own instruction cache. 
A unified or shared instruction cache that is used by all, or many cores of a GPU or CPU has the potential to improve system performance and reduce power consumption\jbs{, however the increased traffic could lead to a higher miss rate, reducing performance and increasing power consumption}. 
\jbs{The previous sentence is really mutilated from my edit. Needs work.}
Unlike data caches which can be read from and written to, an instruction cache is only read, which means there is less possibility for conflict in the cache.
\jbs{I'm not sure less conflict is the right conclusion to read only...} 
In current computer architectures, the instruction caches are independent of one another, however since each processor uses the same set of instructions, it is plausible that a shared instruction cache could introduce non-trivial improvements in performance. 
A single instruction cache could reduce the number of compulsory misses because an instruction previously executed by one streaming multiprocessor may be available for another streaming multiprocessor immediately rather than requiring an additional miss. 

GPGPU-Sim~\cite{bakhodayuan09} is an open-source software package available to simulate GPU architecture. 
It has been validated to be representative of performance on NVIDIA GPUs and provides a reasonable platform for testing alternate highly-parallel computer architectures.
\jbs{This discussion seems more fitted to the methodology part of the results section.}


\es{This is the end of the introduction, but I will leave the stuff below in case someone else needs these examples.}
Here's an example table in Table~\ref{table:example}.

\begin{table}[h!]
  \centering
  \begin{tabular}{|l|l|}
    \hline
    \textbf{Field} & \textbf{Value}\\
    \hline
    \hline
    Page limit & 12 pages\\
    More stuff & 3000\\
  \end{tabular}
  \caption{Example Table}
  \label{table:example}
\end{table}

\textbf{Some bold text}. Now an example list:

\begin{itemize}
\item some item
\item another
\end{itemize}


Example of how to do a URL: \href{http://micro46.stanford.edu}{http://micro46.stanford.edu}


\section{Background}

Instruction cache can have a large impact in a processor's performance. There have been lot of work in the past in improving the performance of instruction cache  CPUs. Techniques such as advanced branch prediction\cite{yeh93} and replacement policies\cite{smith85} have contributed to the high performance of instructin cache in modern CPUs. However, such techniques are not easily applicable to the general-purpose GPUs that are emerging today. 


Modern general-purpose GPUs are designed to have several compute units to maximize throughput. These compute units are compact, and designed to perform simple operations on a large set of data. It is not feasible to implement techniques such as advanced branch prediction on these small compute units, beccause a single GPU unit can have upto a couple hundred of these compute units. To improve the instruction cache performance of GPUs, new solutions need to be developed in ways that maintain the compact and simplistic nature of the compute units. Current GPU architectures are designed with individual L1 instruction cache for each compute unit of the GPU\cite{keckler2011}. It may be possible have multiple compute units share the same instruction cache without incurring serious performance or power penalties.

\section{Shared Instruction Cache Design}

Describe the instruction cache design we propose.


\section{Expected Results}

Describe the results we expect from real simulations. Summarize our
expected findings.

This section should include charts even if the data is entirely made
up. 

\subsection{GPGPU-Sim}
Maybe a subsection on GPGPU-Sim~\cite{bakhodayuan09} and how it could be used.

\section{Conclusions}

The proposed instruction cache designs provide a possible solution to 
increase the efficiency of instruction cache in parallel processors, 
in particular processors whose primary workload includes single programs
with many parallel threads executing the same code. 
The sharing of instruction cache amongst multiple cores can help 
chip designers optimize for less area without compromising performance. 
The extra space recovered from the sharing of instruction cache among 
multiple cores can be repurposed to improve other parts of the 
chip, such as a larger data cache(s) or additional computation units.
We showed results for GPU workloads because they typically exhibit much 
higher levels of parallelism than CPUs while still executing a single 
application.
GPUs traditionally only expose multi-program workload capabilities 
using corse-grained time-sharing among processes.

For future work, the proposed instruction cache design should be 
simulated and tested to verify the affectiveness of the design. 
The different cache architecture parameters that are expected to 
affect the performance of instruction cache design are: number of 
cores sharing the cache, associativity and size of the cache. 
Extensive testing of such parameters should be conducted to determine 
the most optimal instruction cache design for common workloads. 
The proposed design is expected to perform the best on multi-threaded 
applications with a large amount of redundant instructions, so the 
drawbacks of running non-redundant code should also be considered.


\section*{Acknowledgement}
We acknowledge that this idea for reducing the instruction cache area
overhead by aggressively sharing the cache originally came from
previous work on the TRaX
architecture~\cite{spjut08,spjut09,kopta10,spjut12,kopta13,kopta14}.
The TRaX architecture was designed for ray tracing and supports
thousands of threads running the same application but working on
different data.
We would also like to acknowledge the reviews from Konstantin Shkurko and Steven Jacobs.
This work was completed at Harvey Mudd College in Claremont, California.

\bstctlcite{bstctl:etal, bstctl:nodash, bstctl:simpurl}
\bibliographystyle{IEEEtranS}
\bibliography{references}


\end{document}

